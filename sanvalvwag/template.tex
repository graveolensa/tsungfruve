%%% Arctangent Transpetroglyphics Algra Laboratory
%%% LaTeX AMSArt Template FIle 

%%% Xylome Template File

%%%(Today is Boomtime, the 54th day of Discord in the YOLD 3181)
%%% Version 7: Kahōʻāliʻi (Today is Sweetmorn, the 37th day of Bureaucracy in the YOLD 3180)
%%% Version 8: (Today is Boomtime, the 54th day of Discord in the YOLD 3181)
% The machinery of this first section I want to fold into one or two lines of 
% \LaTeX code. 

%% much of the following needs to be folded in elsewhere


%--------------------------------------------------------------------------------------------
\documentclass[reqno,8pt]{atalart}
\usepackage{atal}
\newcommand{\auref}[3][cyan]{\href{#2}{\color{#1}{#3}}}
\newcommand{\Author}{\href{http://owen.maresh.info}{Owen Maresh}}
\newcommand{\Title}{Next Generation Typographic Style Guidelines}
\newcommand{\Keywords}{}
\begin{document}
\title{\Title}
\author{\Author}
%%% Dedication Line
\dedicatory{\begin{CJK}{UTF8}{mj}古池や蛙飛び込む水の音\end{CJK}\\ old pond; a frog leaps in;water’s sound \\-- Bash\={o}}

%%% Abstract
\begin{abstract}
\end{abstract}
\thispagestyle{empty}
\maketitle
%--------------------------------------------------------------------------------------------


\section{First Section}
\subsection{First Subsection}


typographic conventions

for hypergeometric functions:
        aFb's, I want to use Hoefler-Fetish338
        for aWb's (as defined in Gasper),
                Hoefler-Fetish338

${}_a\scrum{\large F}_{b}$

Instead of 

\[
G_{p,q}^{\,m,n} \!\left( \left. \begin{matrix} a_1, \dots, a_p \\ b_1, \dots, b_q \end{matrix} \; \right| \, z \right) = \frac{1}{2 \pi i} \int_L \frac{\prod_{j=1}^m \Gamma(b_j - s) \prod_{j=1}^n \Gamma(1 - a_j +s)} {\prod_{j=m+1}^q \Gamma(1 - b_j + s) \prod_{j=n+1}^p \Gamma(a_j - s)} \,z^s \,ds,
\]

write

\[
\scrum{G}_{p,q}^{\,m,n} \!\left( \left. \begin{matrix} a_1, \dots, a_p \\ b_1, \dots, b_q \end{matrix} \; \right| \, z \right) = \frac{1}{2 \pi i} \int_L \frac{\prod_{j=1}^m \Gamma(b_j - s) \prod_{j=1}^n \Gamma(1 - a_j +s)} {\prod_{j=m+1}^q \Gamma(1 - b_j + s) \prod_{j=n+1}^p \Gamma(a_j - s)} \,z^s \,ds,
\]


\[
_2\scrum{F}_1 (a,b;c;z) = (1-z)^{c-a-b} {}_2F_1 (c-a, c-b;c ; z)
\]

need a good root-of-unity symbol: if we call roots of unity
        zeta and we are dealing with the zeta function
        this is a problem


need a good way of distinguishing the following:

        Euler's phi function
        the golden mean phi
        a phi b summations
        a psi b summations

        ramanujan's phi and psi functions



if I talk about congruence subgroups, let's not call them Gamma(n), without
at least using a different Gamma than the one that we're using to denote
the gamma function.


don't say eta(tau) or eta(q): don't be so fungible about what we're
talking about: or find some way of writing about it that keeps things
straight.

Is this a finite field on $q$-elements, or does this $q$ refer to
all complex numbers of magnitude less than 1?

Lattices should use different characters that Eisenstein series
(so you can talk about E8, the lattice, and E8, the Eisenstein
series at the same time and know what you're talking about)

~                                                  


Li the function, should be denoted with 李, the chinese character

%%% Bibliography

%%% Actually, I'm of the opinion that a nonmodular bibliography is harmful
%%% Hunting for references is an extremely laborious affair. What is "Comp. Rez. 109(3)? anyway"? And
%%% Why isn't the author including a hyperlink? 
%%% And really the metric here is not "oh, I'm defending what I'm saying", but "how fast can the reader find the supporting"
%%% argument to the contention. Having Xanadu would be nice here. Because saying In [Mc 130], Argyllio Argyre argued that the
%%% Phlebotinum Unobtanium Cohomology group S4(1,a) is nonisogenous to the principally polarized abelian variety which is the
%%% hypersheaf subquotient of the moduli stack and the first fundamental group of the Ven Nerstraond epigerbe. Sometimes those
%%% numbers and letters get in the way of the argument. So, drop 'em

\begin{bibdiv}
\begin{biblist}





\end{biblist}
\end{bibdiv}

\end{document}
